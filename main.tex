%%=======================================================================
% !Mode:: "TeX:UTF-8"
% !TEX program  = PdfLaTeX
%%=======================================================================
% 模板名称:hitszbeamer
% 模板版本:V1.1.0
% 模板作者:杨敬轩(Jingxuan Yang)
% 联系作者:yangjingxuan@stu.hit.edu.cn & yanglatex2e@gmail.com
% 模板交流:QQ群:1039392552,加群请备注LaTeX、hitszthesis相关说明
% 模板适用:哈尔滨工业大学(深圳)Beamer模板
% 模板编译:手动编译方法参看 README.md 或 hitszbeamer.pdf
%          编译beamer之前必须编译说明文档:make doc 或双击 makedoc.bat
%          编译说明文档同时分离出四个样式文件 *hitszbeamer.sty
%          GNU make 工具:make beamer
%          Windows批处理脚本:双击 makebeamer.bat 自动编译论文
%          更多编译细节详见说明文档:hitszbeamer.pdf
% 更新时间:2022/05/20
% 模板帮助:请**务必务必务必**阅读 hitszbeamer.pdf 说明文档,文档查看方法:
%          下载模板文件夹里就有,如果是从CTAN上安装更新本模板,则通过
%          cmd 命令行:texdoc hitszbeamer
%          推荐前往模板的 GitHub 仓库获取最新文件,地址:
%          https://github.com/YangLaTeX/hitszbeamer
%%=======================================================================

% 设置文档类别为 <beamer>,<compress>选项含义为尽量压缩导航栏
\documentclass[compress]{beamer}

% 使用 <hitszbeamer> 主题
\usetheme{hitszbeamer}

% 开始写文章
\begin{document}

% 图片存放路径
\graphicspath{{figures/}}

% 封面信息
\title[报告标题]{报告标题\\[2mm] 中期答辩}
\author[杨敬轩]{学生:杨敬轩\\[5mm] 导师:XX\ 教授}
\institute[哈尔滨工业大学(深圳)]{\small  哈尔滨工业大学(深圳)}
\date{\small \vskip -10pt \today}

% 标题页
\begin{frame}
		\maketitle
\end{frame}

% 目录页
\section*{目录}
\frame{
  \frametitle{\secname}
  \tableofcontents[hideallsubsections]
}

\section{研究内容及进度}

\subsection{课题主要研究内容}

\begin{frame}{课题主要研究内容}
  \begin{figure}
    \includegraphics[width=0.6\linewidth]{hitlogo}
    \caption{课题主要研究内容}
  \end{figure}
\end{frame}

\subsection{进度介绍}

\begin{frame}{进度介绍}
  \begin{figure}
    \includegraphics[width=0.6\linewidth]{hitlogo}
    \caption{进度介绍}
  \end{figure}
\end{frame}

\section{研究工作及成果}

\begin{frame}{已完成的研究工作及成果}
  \begin{block}{已完成的研究工作简介}
    \begin{itemize}
      \setlength{\itemsep}{6pt}
      \item XXXX
      \item XXXX
      \item XXXX
      \item XXXX
      \item XXXX
      \item XXXX
    \end{itemize}
  \end{block}
\end{frame}

\subsection{研究工作一}

\begin{frame}{研究工作一}
  \begin{block}{定义某个系数}
    $$
    k: \left[ -\pi , \pi \right] \rightarrow \left[ 0, 1 \right]
    $$
  \end{block}
\end{frame}

\subsection{研究工作二}

\begin{frame}{研究工作二}

\end{frame}

\subsection{研究工作三}

\begin{frame}{研究工作三}

\end{frame}

\subsection{研究工作四}

\begin{frame}{研究工作四}

\end{frame}

\section{后期工作与安排}

\subsection{后期研究工作}

\begin{frame}{后期研究工作}
  \begin{block}{后期研究工作}
    \begin{itemize}
      \setlength{\itemsep}{6pt}
      \item XXXX
      \item XXXX
      \item XXXX
      \item XXXX
    \end{itemize}
  \end{block}
\end{frame}

\subsection{进度安排}

\begin{frame}{进度安排}
  \begin{block}{进度安排}
    \begin{itemize}
      \setlength{\itemsep}{6pt}
      \item XXXX\cite{liu2016}
      \item XXXX\cite{ren2010}
      \item XXXX
      \item XXXX
    \end{itemize}
  \end{block}
\end{frame}

\section{问题与解决方案}

\begin{frame}{问题与解决方案}
  \begin{block}{问题}
    \begin{itemize}
      \item XXXX
    \end{itemize}
  \end{block}
  \begin{block}{对应解决方案}
    \begin{itemize}
      \item XXXX
    \end{itemize}
  \end{block}
\end{frame}

\section{按时完成可能性}

\begin{frame}{按时完成可能性}
  \begin{block}{按时完成可能性}
    \begin{itemize}
      \setlength{\itemsep}{6pt}
      \item XXXX\cite{Chen1992}
      \item XXXX\cite{Gravagne2003}
      \item XXXX\cite{xin1994}
      \item XXXX\cite{zhai2015}
    \end{itemize}
  \end{block}
\end{frame}

\bibliographystyle{hitszbeamer}

\begin{frame}[allowframebreaks]{参考文献}
  \bibliography{reference}
\end{frame}

\section{Q\&A}

\begin{frame}{\secname~ }
	\begin{center}
 		\huge {That's all. Thank you!}\\
		\vspace{1cm}
		\huge {Q\&A}
	\end{center}
\end{frame}

% 结束文档撰写
\end{document}
